%=============================================================================
%                           DOCUMENT BODY CONTENT
%=============================================================================
%
% QUICK REFERENCE - Available Commands:
%
%   \parasection{Title}      Numbered section heading (1. Title, 2. Title, etc.)
%   \begin{subpara}          Lettered sub-points (a., b., c.)
%   \begin{subsubpara}       Numbered sub-sub-points ((1), (2), (3))
%
%   \enclref{1}              Clickable "Enclosure (1)" - jumps to enclosure 1
%   \encl{1}                 Clickable "(1)" - short form, jumps to enclosure 1
%
%-----------------------------------------------------------------------------
% COPY/PASTE TEMPLATES:
%-----------------------------------------------------------------------------
%
% Simple paragraph section:
%   \parasection{Section Title}
%   Your paragraph text here.
%
% Section with bullet points:
%   \parasection{Section Title}
%   \begin{subpara}
%   \item First point.
%   \item Second point.
%   \end{subpara}
%
% Nested sub-points:
%   \begin{subpara}
%   \item Main point:
%       \begin{subsubpara}
%       \item Detail one.
%       \item Detail two.
%       \end{subsubpara}
%   \end{subpara}
%
% Enclosure references (clickable links):
%   See \enclref{1} for details.        -> "See Enclosure (1) for details."
%   Per reference \encl{3}, the...      -> "Per reference (3), the..."
%
%=============================================================================


%-----------------------------------------------------------------------------
% 1. PURPOSE
%-----------------------------------------------------------------------------
\parasection{Purpose}

To request administrative support for upcoming training operations per references (a) and (b).


%-----------------------------------------------------------------------------
% 2. BACKGROUND
%-----------------------------------------------------------------------------
\parasection{Background}

\begin{subpara}

\item \textbf{Overview.} This section provides context for the request. Include relevant history, previous actions, and any pertinent information that supports the purpose of this correspondence.

\item \textbf{Current Situation.} Describe the current state of affairs that necessitates this correspondence. Be specific about dates, locations, and personnel involved as appropriate.

\item \textbf{Previous Actions.} List any prior actions taken to address this matter, including dates and outcomes.

\end{subpara}


%-----------------------------------------------------------------------------
% 3. DISCUSSION
%-----------------------------------------------------------------------------
\parasection{Discussion}

\begin{subpara}

\item \textbf{Analysis.} Provide detailed analysis of the situation. Include:
    \begin{subsubpara}
    \item First point of analysis with supporting details.
    \item Second point of analysis with supporting details.
    \item Third point of analysis with supporting details.
    \end{subsubpara}

\item \textbf{Options Considered.} If applicable, discuss alternatives that were considered and explain why the recommended course of action is preferred.

\item \textbf{Resource Requirements.} Identify any resources (personnel, equipment, funding) required to implement the recommendation.

\end{subpara}


%-----------------------------------------------------------------------------
% 4. RECOMMENDATION
%-----------------------------------------------------------------------------
\parasection{Recommendation}

Request approval for the following actions:

\begin{enumerate}[label=\alph*.]
    \item First recommended action.
    \item Second recommended action.
    \item Third recommended action.
\end{enumerate}


%=============================================================================
% END OF BODY CONTENT
%=============================================================================
